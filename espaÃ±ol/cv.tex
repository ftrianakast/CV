\documentclass[spanish]{resume} % Use the custom resume.cls style
\usepackage[left=0.75in,top=0.6in,right=0.75in,bottom=0.6in]{geometry} % Document margins
\usepackage[spanish]{babel}
\usepackage{hyperref}

\name{Felipe Triana Casta\~neda} % Your name
\address{Cl. 131a #9-59 \\ Bogot\'a, Colombia} % Your address
%\address{\url{https://ftrianakast-cedar-14.herokuapp.com/}}
\address{(+57)~$\cdot$~302~$\cdot$~8454916 \\ ftrianakast@gmail.com} % Your phone number and email


\begin{document}

%----------------------------------------------------------------------------------------
% SUMMARY SECTION
%----------------------------------------------------------------------------------------

\begin{rSection}{Resumen}

Ingeniero de software con buenas habilidades comunicativas y de trabajo en grupo. Entusiasta en temas de arquitectura de software, ciencia de datos, programci\'on funcional, sistemas distribuidos y teor\'ia de categor\'ias. Como ingeniero de software capaz de definir y llevar a cabo arquitecturas basadas en m\'ultiples frameworks y lenguajes de programaci\'on, y como entusiasta de ciencia de datos estoy familiarizado con t\'ecnicas estad\'isticas, de an\'alisis de redes y de machine learning.

\end{rSection}

%----------------------------------------------------------------------------------------
% INTERESTS SECTION
%----------------------------------------------------------------------------------------
\begin{rSection}{Intereses}

\item Sistemas distribuidos.
\item Programaci\'on funcional (Scala, Haskell).
\item Teor\'ia de Categor\'ias.
\item Ciencia de datos.
\item Probabilidad y estad\'istica.

\end{rSection}


%----------------------------------------------------------------------------------------
% WORK EXPERIENCE SECTION
%----------------------------------------------------------------------------------------
\begin{rSection}{Experiencia}

\begin{rSubsection}{eBay}{Mar 2022 - Presente}{MTS 1, Software Engineer}{Berl\'in}
\item Lider\'e m\'ultiples proyectos para soportar la plataforma de Rewards en Europa.
\item Creaci\'on de directivas t\'ecnicas y directrices de arquitectura alrededor de la plataforma de Rewards.
\item Soporte y creaci\'on de microservicios utilizando Java y Scala para el soporte de la plataforma de Rewards en Alemania.

\begin{rSubsection}{eBay}{Nov 2018 - Mar 2022}{Senior Software Engineer}{Berl\'in}
\item Creaci\'on de una plataforma para el manejo de Rewards al interior de eBay Alemania.
\item El sistema de Rewards soporta millones de usuarios en Europa y fue creado usando microservicios basados en Scala y Java.
\item Soporte t\'ecnico de varias iniciativas a nivel global incluyendo pagos, carrito y checkout.
\item Trabaj\'e con un equipo internacional en la ciudad de Berl\'in coequipando con m\'ultiples individuos de diferentes culturas.
\item Los microservicios fueron creados utilizando la infraestructura interna de eBay.


\begin{rSubsection}{Zalando}{May 2017 - Nov 2018}{Software Engineer}{Berl\'in}
\item Trabaj\'e en un proyecto llamado Collabary al interior de Zalando. Collabary es una plataforma que conecta influenciadores con empresas que quieran hacer promociones: permite la b\'usqueda de influenciadores, el manejo de campa\~nas y el presupuesto asociado a ellas en Europa.
\item Ingeniero de software backend usando Java y NodeJS.
\item En t\'erminos de Java trabaj\'e utilizando Java 8, Spring Boot y Hibernate.
\item En t\'erminos de NodeJS trabaj\'e utilizando ExpressJS y Sequelize.
\item Para modelaje de bases de datos utilic\'e Postgres.
\item En t\'erminos de infraestructura utilic\'e AWS, Docker y una herramienta interna para el manejo de microservicios.
\item Trabaj\'e con un equipo internacional en la ciudad de Berl\'in coequipando con m\'ultiples individuos de diferentes culturas.


\end{rSubsection}


\begin{rSubsection}{Prodigious Latin America}{Marzo 2015 - Abril 2017}{Senior Software Engineer}{Bogot\'a D.C| San Francisco}
\item Trabajé en tres proyectos como referente técnico: Hewllett Packard, proyecto USANA, proyecto Renault-Nissan.
\item AEM developer: componentes, workflows y servicios OSGi.
\item Diseñador de webservices y desarrollador sobre la plataforma Apache Sling.  
\item Ingeniero Backend usando lenguaje Java.
\item Experiencia con un equipo internacional de desarrolladores.
\end{rSubsection}


\begin{rSubsection}{Aentropico SAS}{Mayo 2014 - Octubre 2014}{CTO}{Bogot\'a D.C | R\'io de Janeiro}
\item L\'ider tecnol\'ogico de la organizaci\'on. Encargado de dar las directrices de tecnolog\'ia que soportan el negocio.
\item Dise\~no de un API Rest para el consumo de la plataforma de an\'alisis de datos.
\item Ingenier\'ia de software utilizando diferentes lenguajes de programaci\'on para llevar acabo la arquitectura mencionada: NodeJS, R y Java.
\item Desarrollador frontend utilizando Javascript (AngularJS Framework), CSS3 (preprocesador Stylus), HTML5 (motor de templating Jade).
\item Dise\~no de base de datos utilizando tecnolog\'ia NoSQL (MongoDB).
\item Ingeniero de infraestructura encargado del despliegue (AWS, Docker, Heroku), de la seguridad (OAuth 2.0) y de la escalabilidad de la plataforma.
\item Ingeniero de requerimientos. A partir de requerimientos de negocio discernir requerimientos funcionales.
\item Scrum master. Planeaci\'on de sprints y responsable de fechas de entrega.
\end{rSubsection}

%------------------------------------------------

\begin{rSubsection}{Aentropico SAS}{Agosto 2013 - Mayo 2014}{Full Stack Developer}{Bogot\'a D.C | R\'io de Janeiro}
\item Dise\~no arquitectural de una plataforma que provee herramientas de An\'alisis Predictivo (machine learning) y de algor\'itmica avanzada para cualquier tipo de negocios.
\item Dise\~no de un API Rest para el consumo de la plataforma de an\'alisis.
\item Ingenier\'ia de software utilizando diferentes lenguajes de programaci\'on para llevar acabo la arquitectura mencionada: NodeJS, R y Java.
\item Desarrollador frontend utilizando Javascript (AngularJS Framework), CSS3 (preprocesador Stylus), HTML5 (motor de templating Jade).
\item Dise\~no de base de datos utilizando tecnolog\'ia NoSQL (MongoDB).
\item Ingeniero de infraestructura encargado del despliegue (AWS, Docker, Heroku), de la seguridad (OAuth 2.0) y de la escalabilidad de la plataforma.
\item Ingeniero de requerimientos. A partir de requerimientos de negocio discernir requerimientos funcionales.
\item Scrum master. Planeaci\'on de sprints y responsable de fechas de entrega.
\end{rSubsection}

%------------------------------------------------

\begin{rSubsection}{IBM}{Agosto 2012 - Diciembre 2012}{Analista T\'ecnico}{Bogot\'a D.C}
\item Ingeniero de integraci\'on de aplicaciones encargado del levantamiento de requerimientos de integraci\'on e implementaci\'on de los mismos para una empresa del sector bancario. El trabajo estuvo enmarcado dentro de una metodolog\'ia tradicional de desarrollo certificada nivel 5 dentro del modelo de madurez CMMI.
\item Implementaci\'on de servicios de integraci\'on utilizando la plataforma Websphere Message Broker.
\item Implementaci\'on de adaptadores Java para aplicaciones legado.
\end{rSubsection}

%------------------------------------------------

\begin{rSubsection}{SoftOne SAS}{Junio 2012 - Agosto 2012}{Desarrollador Freelancer}{Bogot\'a D.C}
\item Desarrollo de servicios web sobre el framework JEE para el manejo de los canales IVR y Kioskos de una empresa financiera.
\item Desarrollo de una aplicaci\'on m\'ovil sobre la plataforma J2ME para la distribuci\'on de pedidos.
\item Los dos proyectos estuvieron enmarcados dentro de una metodolog\'ia \'agil de desarrollo (Scrum).
\end{rSubsection}

\end{rSection}


%----------------------------------------------------------------------------------------
% EDUCATION SECTION
%----------------------------------------------------------------------------------------

\begin{rSection}{Educaci\'on}

{\bf Universidad de los Andes, Bogot\'a D.C} \hfill {\em Agosto 2013} \\
Ingenier\'ia de Sistemas \& Computaci\'on \\
Tesis SkinHealth, un sistema para la detecci\'on de enfermedades de la piel utilizando la Ontolog\'ia Ontoderm.

{\bf Colegio Seminario San Juan Ap\'ostol, Facatativ\'a} \hfill {\em 2008} \\
Bachiller Acad\'emico.

\end{rSection}


%----------------------------------------------------------------------------------------
% PERSONAL INITIATIVES
%----------------------------------------------------------------------------------------
\begin{rSection}{Iniciativas Personales}

\begin{rSubsection}{Bogot\'a Lambda}{January 2016}{Coorganizador}{Bogot\'a D.C}
\item Coorganizador y promotor de un meetup relacionado con concepts de programaci\'on funcional.
\item Conferencista en m\'ultiples meetups sobre conceptos de programaci\'on funcional usando Scala y Haskell.
\item Dise\~nador de ejercicios para practicar conceptos de proframaci\'on funcional usando Scala y Haskell.
\end{rSubsection}

\end{rSection}

%----------------------------------------------------------------------------------------
% TECHNICAL STRENGTHS SECTION
%----------------------------------------------------------------------------------------

\begin{rSection}{Habilidades t\'ecnicas}

\begin{tabular}{ @{} >{\bfseries}l @{\hspace{6ex}} l }
Arquitectura de Software & SOA (servicios Rest y SOAP), DDD, MDD, patrones de dise\~no \\
Lenguajes de programaci\'on & Java, Scala, Pyhton, Javascript, Haskell \\
Librer\'as & Scalaz \\ 
Frameworks y plataformas & ZIO, Spring, JEE, NodeJS, ExpressJS \\
Frontend & AngularJS, BackboneJS, CSS3, HTML5, Jade, Stylus, Grunt, Gulp \\
Protocolos \& APIs & XML, JSON, SOAP, REST \\
Databases & MySQL, PostgreSQL, MongoDB, Redis, Neo4j \\
Tools & Maven, SBT, Npm, Git, SVN
\end{tabular}

\end{rSection}

%----------------------------------------------------------------------------------------
% PROJECT MANAGEMENT SKILLS
%----------------------------------------------------------------------------------------
\begin{rSection}{Habilidades de administraci\'on de proyectos}

\begin{tabular}{ @{} >{\bfseries}l @{\hspace{6ex}} l }
Metodolog\'ias & Scrum, Kanban. \\
Comunicaci\'on & Buena expresi\'on oral y escrita.
\end{tabular}

\end{rSection}


%----------------------------------------------------------------------------------------
% LANGUAGES
%----------------------------------------------------------------------------------------
\begin{rSection}{Idiomas}

\begin{tabular}{ @{} >{\bfseries}l @{\hspace{6ex}} l }
Ingl\'es & Competencia profesional completa. \\
Alem\'n & Competencia b\'asica.
Espa\~nol & Nativo.
\end{tabular}

\end{rSection}

%----------------------------------------------------------------------------------------
% RECONOCIMIENTOS
%----------------------------------------------------------------------------------------
\begin{rSection}{Reconocimientos}

{\bf Beca quiero estudiar Uniandes} \hfill {\em 2009} \\
Universidad de los Andes, Bogot\'a D.C.
Otorgada a los ICFES m\'as sobresalientes del pa\'is.

{\bf Mejor Bachiller Acad\'emico COLSEM} \hfill {\em 2008} \\
Colegio Seminario San Juan Ap\'ostol.

{\bf Speaker Javascript Conference, Medell\'in Colombia} \hfill {\em Octubre 2013} \\
Conferencista invitado en la conferencia anual de Javascript representando a Aentr\'opico SAS.
Tema: explorador de visualizaciones, una danza entre nodeJS y D3.js.


\end{rSection}


%----------------------------------------------------------------------------------------
% CURSOS
%----------------------------------------------------------------------------------------
\begin{rSection}{Cursos - Conferencias}

{\bf Coursera, Yale} \hfill {\em May 2021} \\
Financial Markets
Este curso provee informaci\'on extensa para gestionar el riesgo y las finanzas, y provee elementos de finanzas conductuales que permiten entender como funciona la industria de valores, seguros y banca.

{\bf Coursera, Standford} \hfill {\em Abril 2016} \\
Machine Learning.
Este curso provee una introducción extensa al área de Machine Learning, datamining y reconocimiento de patrones estadísticos. Entre los temas a tratar se pueden encontrat: (i) Supervised learning (parametric/non-parametric algorithms, support vector machines, kernels, redes neuronales). (ii) Unsupervised learning (clustering, dimensionality reduction, recommender systems, deep learning). (iii) Mejores practicas en machine learning(bias/variance theory; procesos de innovación en machine learning e inteligencia artificial).

{\bf Coursera, \'Ecole Polytechnique F\'ed\'erale de Lausanne} \hfill {\em Mayo 2015} \\
Principles of Reactive Programming.
Elementos claves para escribir programas reactivos de una manera componible y la manera de aplicar estos elementos en la construcción de message-driven systems que son escalables y resilentes. 

{\bf Coursera, Escuela Polit\'ecnica Federal de Lausana} \hfill {\em Septiembre 2014} \\
Functional Programming Principles in Scala.
Curso sobre temas introductorios y avanzados de programaci\'on funcional usando Scala.

{\bf edX, Universidad de Berkeley, California} \hfill {\em Septiembre 2010} \\
Artifitial Intelligence.
Principios de inteligencia artificial.

\end{rSection}


\end{document}