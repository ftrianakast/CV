%!TEX TS-program = xelatex
\documentclass[]{friggeri-cv}
\addbibresource{bibliography.bib}

\begin{document}
\header{felipe}{triana}
       {ingeniero de software}


% In the aside, each new line forces a line break
\begin{aside}
  \section{contacto}
    Calle 6B No 9-26
    Madrid Cundinamarca
    Colombia
    ~
    (0057) 3108187426
    ~
    \href{mailto:fastalfe2006@gmail.com}{fastalfe2006@gmail.com}
    \href{https://github.com/ftriana3185}{github://ftriana3185}
    \href{http://co.linkedin.com/pub/felipe-triana-casta%C3%B1eda/55/bb9/4a}{linkedIn://felipeTriana}
  \section{Idiomas}
    inglés nivel avanzado
    portugués nivel básico
  \section{Programación}
    Java
    JavaScript,Scala
    R,Go
    CSS3 \& HTML5
\end{aside}

\section{perfil}

Ingeniero de software con buenas habilidades comunicativas y de trabajo en grupo, y con gran capacidad de autoaprendizaje. Entusiasta en temas de arquitectura de software y ciencia de datos. Como ingeniero de software capaz de definir y llevar a cabo arquitecturas basadas en múltiples frameworks y lenguajes de programación, y como entusiasta de ciencia de datos estoy familiarizado con técnicas estadísitca, de análisis de redes y de machine learning.   

\section{intereses}

ciencia de datos, arquitectura de software, ingeniería de software, sistemas distribuidos, computación gráfica, programación funcional, arquitectura empresarial, estadística, matemáticas y emprendimiento 

\section{educación}

\begin{entrylist}
  \entry
    {2009-2013}
    {Ingeniero de sistemas y computación}
    {Universidad de los Andes, Bogotá D.C}
    {Tesis, SkinHealth: un sistema para la autodetección de enfermedades dermatológicas usando la ontología OntoDerm}
  \entry
    {2008}
    {Bachiller académico}
    {Colegio Seminario San Juan Apóstol, Facatativá}
    {Mejor bachiller académico}
\end{entrylist}


\section{experiencia}

\begin{entrylist}
  \entry
    {2014-Act}
    {Aentrópico, Bogotá D.C, Rio de Janeiro}
    {Full Stack Developer}
    {\emph{Líder en la construcción de una plataforma para la distribución de servicios de análisis de grandes datos para gobierno, empresas de retail, petroleras y empresas del sector financiero. Encargado de definir arquitecturas y de llevarlas a cabo. El trabajo está guiado por una metodología ágil de desarrollo: Scrum y basado ampliamente en las tecnologías NodeJS, R language y el framework front-end AngularJS}}
  \entry
    {2013-2014}
    {IBM de Colombia, Bogotá D.C}
    {Internship, Ingeniero de integración}
    {\emph{Ingeniero de integración de aplicaciones encargado del levantamiento de requerimientos y desarrollo de los mismos sobre la plataforma WebSphere Message Broker de IBM para una empresa del sector bancario. El trabajo estuvo enmarcado dentro de una metodología tradicional de desarrollo de software certificada nivel 5 dentro del modelo de madurez CMMI}}
  \entry
    {2011}
    {SoftOne SAS, Bogotá D.C}
    {Ingeniero de Software}
    {\emph{Ingeniero de software durante 3 meses. Desarrollo de una aplicación móvil para la distribución de pedidos sobre la plataforma J2ME y de una aplicación empresarial sobre J2EE para el manejo de los canales IVR y Kioskos de una empresa financiera}}
\end{entrylist}


\section{logros y reconocimientos}

\begin{entrylist}
  \entry
    {2009}
    {Beca Quiero Estudiar}
    {Universidad de los Andes, Bogotá}
    {Otorgada para los estudios de Ingeniería de Sistemas y computación}
  \entry
    {2008}
    {Mejor Bachiller Académico}
    {Colegio Seminario San Juan Apóstol, Facatativá}
    {Otorgado por José Antonio Suárez Alarcón, Rector}
\end{entrylist}


\section{cursos | congresos}

\begin{entrylist}
  \entry
    {2014}
    {Functional Programming Principles in Scala}
    {EPFL, Zürich Suiza}
    {Curso sobre programación funcional en el lenguaje Scala}
  \entry
    {2013}
    {Speaker en Javascript Conference, Medellín Colombia}
    {Aentrópico}
    {Conferencista en la conferencia anual de Javascript representando a Aentrópico. El tema: Explorador de visualizaciones: una danza entre nodejs y D3.js}
  \entry
    {2012}
    {Artificial Intelligence}
    {Universidad de Berkeley, California}
    {Curso introductorio a la inteligencia artificial dictado a través del edX}
\end{entrylist}


\section{tecnologías}



%%% This piece of code has been commented by Karol Kozioł due to biblatex errors. 
% 
%\printbibsection{article}{article in peer-reviewed journal}
%\begin{refsection}
%  \nocite{*}
%  \printbibliography[sorting=chronological, type=inproceedings, title={international peer-reviewed conferences/proceedings}, notkeyword={france}, heading=subbibliography]
%\end{refsection}
%\begin{refsection}
%  \nocite{*}
%  \printbibliography[sorting=chronological, type=inproceedings, title={local peer-reviewed conferences/proceedings}, keyword={france}, heading=subbibliography]
%\end{refsection}
%\printbibsection{misc}{other publications}
%\printbibsection{report}{research reports}

\end{document}
